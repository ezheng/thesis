%%%%%%%%%%%%%%%%%%%%%%%%%%%%%%%%%%%%%%%%%%%%%%%%%%%%%%%%%%%%
%%%%%%%%%%%%%%%%%%%%%%%%%%%%%%%%%%%%%%%%%%%%%%%%%%%%%%%%%%%%
\chapter{Related work}

3D reconstruction from 2D imagery has been studied intensively and extensively by many researchers in the computer vision community. In this section, I first review the works for camera parameter estimation, and then survey the works relating to static and dynamic object reconstruction respectively.

\section{Camera parameter estimation}
Camera parameters are typically considered as an prerequisite for 3D reconstruction, since they provide the geometric relationship among different cameras.
The camera parameters include the  internal (intrinsic) parameter and external (extrinsic) parameter. While the internal parameters have focal length, principle point, skew parameter, and radial distortion of an camera, the external parameters describe a camera's rotation and translation relative to a global coordinate system \cite{Hartley2004}. 

Given the importance of camera parameters in the task of 3D reconstruction, lots of works have been done for camera calibration -- a process determining camera parameters. 
Earlier works require a calibration object such as a planar checkerboard to be seen by the cameras \cite{conf/cvpr/SturmM99,zhang2000flexible,caltoolbox}. 
This 


