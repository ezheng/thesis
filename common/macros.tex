% !TEX root = ../thesis_heinly.tex

\newcommand{\set}[1]{$\mathbb{#1}$}

% tables
\newcommand{\thickhline}{\noalign{\hrule height 1.5pt}}
\newcommand{\fixcell}[1]{\raisebox{-1pt}{#1}}

% citation style
% default: cite with (Name, year)
\renewcommand{\cite}{\citep}

% common abbreviations
\newcommand{\eg}{{\it e.g.~}}
\newcommand{\ie}{{\it i.e.~}}
\newcommand{\etc}{{\it etc.}\xspace}
\newcommand{\etal}{\emph{et~al}\mbox{.}\xspace}

\newcommand{\xth}{\ensuremath{^{\text{th}}}\xspace}
%\newcommand{\fst}{\ensuremath{^{\text{st}}}\xspace}

\newcommand{\vs}{{vs\mbox{.}}\xspace}

% common Math notation
\newcommand{\NAT}[0]{\mathbb{N}\xspace}
\newcommand{\fun}[1]{\mathit{#1}} % typeset as function name
\newcommand{\setsize}[1]{\left| #1 \right|}
\newcommand{\setdef}[2]{\left\{ #1 \ \left|\  #2\right.\right\}}
\newcommand{\dispsum}[0]{\displaystyle\sum}

% matrix
\def\by{\! \times \!}

\newcommand{\defeq}[0]{\triangleq}
\renewcommand{\mod}{\operatorname{mod}}

% time units
\newcommand{\mus}[0]{\ensuremath{\mu s}\xspace}
\newcommand{\us}[0]{\ensuremath{\mu s}}
\newcommand{\ms}[0]{\ensuremath{\fun{ms}}\xspace}

% algorithm names
\newcommand{\kwfont}[1]{\textsf{#1}\xspace} %\small
% variable name
\newcommand{\var}[1]{\ensuremath{{\fun{#1}}}\xspace} %\small

%http://hstuart.dk/2007/08/03/programming-latex-%E2%80%94-writing-commands/
\newcommand{\mkkw}[2]{
	\newcommand{#1}[0]{\kwfont{#2}}
}

% fancy symbols and functions
\newcommand{\Alg}[0]{{\mathcal A}}
\newcommand{\Test}[0]{{\mathcal T}}
\newcommand{\Mach}[0]{{\mathcal M}}

\newcommand{\usum}[0]{u_{\mathrm{sum}}}
\newcommand{\umax}[0]{u_{\mathrm{max}}}
\newcommand{\umin}[0]{u_{\mathrm{min}}}
\newcommand{\utop}[0]{u_{\mathrm{top}}}

\newcommand{\esum}[0]{e_{\mathrm{sum}}}
\newcommand{\emax}[0]{e_{\mathrm{max}}}
\newcommand{\emin}[0]{e_{\mathrm{min}}}
\newcommand{\etop}[0]{e_{\mathrm{top}}}

\newcommand{\dsum}[0]{\delta_{\mathrm{sum}}}
\newcommand{\dmax}[0]{\delta_{\mathrm{max}}}
\newcommand{\dmin}[0]{\delta_{\mathrm{min}}}
\newcommand{\dtop}[0]{\delta_{\mathrm{top}}}

\newcommand{\prio}[0]{\mathsf Y}
\newcommand{\eprio}[0]{\mathsf y}

% src code
\newcommand{\src}[1]{\textsf{\small #1}\xspace}

% references
\newcommand{\alref}[1]{Algorithm~\ref{al:#1}\xspace}
\newcommand{\stpref}[1]{Step~\ref{stp:#1}\xspace}
\newcommand{\stprefs}[2]{Steps~\ref{stp:#1} and~\ref{stp:#2}\xspace}
\newcommand{\chref}[1]{Chapter~\ref{ch:#1}\xspace}
\newcommand{\chrefs}[2]{Chapters~\ref{ch:#1} and~\ref{ch:#2}\xspace}
\newcommand{\secref}[1]{Section~\ref{sec:#1}\xspace}
\newcommand{\secrefs}[2]{Sections~\ref{sec:#1} and~\ref{sec:#2}\xspace}
\newcommand{\figref}[1]{Figure~\ref{fig:#1}\xspace}
\newcommand{\figrefs}[2]{Figures~\ref{fig:#1} and~\ref{fig:#2}\xspace}
\newcommand{\figrefi}[2]{Figure~\ref{fig:#1}(#2)\xspace}
\newcommand{\tabref}[1]{Table~\ref{tab:#1}\xspace}
\newcommand{\tabrefs}[2]{Tables~\ref{tab:#1} and~\ref{tab:#2}\xspace}
\newcommand{\lemref}[1]{Lemma~\ref{lem:#1}\xspace}
\newcommand{\thmref}[1]{Theorem~\ref{thm:#1}\xspace}
\newcommand{\defref}[1]{Definition~\ref{def:#1}\xspace}
\newcommand{\exref}[1]{Example~\ref{ex:#1}\xspace}
\newcommand{\equref}[1]{Equation~(\ref{eq:#1})\xspace}
\newcommand{\inequref}[1]{Inequality~(\ref{eq:#1})\xspace}
\newcommand{\lstref}[1]{Listing~\ref{lst:#1}\xspace}
\newcommand{\pref}[1]{page~\pageref{p:#1}\xspace}

% parameters

\newcommand{\pacc}[0]{\var{pacc}}
\newcommand{\wratio}[0]{\var{wratio}}

% special footnotes

% from http://help-csli.stanford.edu/tex/latex-footnotes.shtml
\long\def\symbolfootnote[#1]#2{\begingroup%
\def\thefootnote{\fnsymbol{footnote}}\footnote[#1]{#2}\endgroup}

% Theorems, etc.

\newtheoremstyle{mylemthm}% hnamei 
        {6pt}% hSpace abovei 
        {3pt}% hSpace belowi 
        {\slshape}% hBody fonti 
        {}% hIndent amounti1
        {\bfseries}% hTheorem head fonti 
        {.}% hPunctuation after theorem headi 
        {.5em}% hSpace after theorem headi2
        {}% hTheorem head spec (can be left empty, meaning `normal')i

\theoremstyle{mylemthm}

\newtheorem{theorem}{Theorem}[chapter]
\newtheorem{lemma}{Lemma}[chapter]

%\theoremstyle{definition}

\newtheoremstyle{mydef}% hnamei 
        {3pt}% hSpace abovei 
        {3pt}% hSpace belowi 
        {\normalfont}% hBody fonti 
        {}% hIndent amounti1
        {\bfseries}% hTheorem head fonti 
        {.}% hPunctuation after theorem headi 
        {.5em}% hSpace after theorem headi2
        {\thmname{#1} \thmnumber{#2}\thmnote{#3}}% hTheorem head spec (can be left empty, meaning `normal')i

\theoremstyle{mydef}


%% Flush words right at end of paragraph.
%% From: http://tex.stackexchange.com/questions/16330/hfill-after-linebreak
\newcommand\rightparend[1]{{%
      \unskip\nobreak\hfil\penalty50
      \hskip2em\hbox{}\nobreak\hfil\textbf{#1}%
      \parfillskip=0pt \finalhyphendemerits=0 \par}}


\newtheorem{definition}{Definition}[chapter]
\newtheorem{xxexample}{Example}[chapter]

%% "inherent" from xxexample, but place box at the end of example.
\newenvironment{example}{
\begin{xxexample}
}{
\rightparend{$\Diamond$}
\end{xxexample}
}
% \qed   \sqbullet \blackdiamond \vartriangleleft


%
\newcommand{\wrt}{with respect to }
\newcommand{\jost}{joint object class sequencing and trajectory triangulation}
\newcommand{\JOST}{Joint Object Class Sequencing and Trajectory Triangulation}
\newcommand{\oct}{object class trajectory}
\newcommand{\OCT}{Object Class Trajectory}

\newcommand\allX{\mathbb X}
\newcommand\oneX{\mathbf X}
\newcommand\shape{\mathbf S}

\newcommand\allT{\mathbb W}
\newcommand\oneT{\mathbf W}

\newcommand\allC{\mathbb C}
\newcommand{\oneC}{\mathbf C}

\newcommand{\allr}{\mathbbm r}
\newcommand{\oner}{\mathbf r}

\newcommand{\alld}{\mathbbm d}
\newcommand{\oned}{\mathbf d}

